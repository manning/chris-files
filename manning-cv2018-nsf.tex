\documentclass[11pt,letterpaper]{article}

\usepackage{palatino}
\usepackage{cv}
\usepackage{textcomp}

\textheight=9in
\textwidth=6.5in
\oddsidemargin=0in
% \advance\topmargin -0.6in

\def\url#1{{\small\sf #1}}
% \def\enumsquare{\raise1pt\hbox{$\scriptstyle\blacksquare$}}
% \renewcommand{\labelitemi}{\enumsquare}

%\lines
%\refsonrequest
%\dated
\centername
\namesize{\Large}
\homeaddresstitlefalse

\begin{document}
\vspace*{-0.75cm}

\name{\bf Christopher David Manning}
\businessaddress{Department of Computer Science\\
        Gates Building 2A, 353 Serra Mall\\
        Stanford CA 94305-9020\\
        USA}

\homeaddress{Phone: +1 (650) 723-7683\\
	Fax:  +1 (650) 725-1449 \\
	Email: \small\sf manning@cs.stanford.edu \\
        Web: \small\sf https://nlp.stanford.edu/\texttildelow manning/}

\newcategory{Professional Preparation}
\newdatedcategory{Appointments}
\newcategory{Collaborators and Other Affiliations}
\newcategory{Products (most closely related)}
\newcategory{Products (other significant)}
\newcategory{Synergistic Activities}
\newcategory{Membership in Professional Organizations}

\begin{vita}
\begin{Professional Preparation}
\item The Australian National University. B.A.\ (Hons) with First Class Honors and University
Medal in Linguistics. Additional majors in Mathematics and Computer
Science. 1989.

\item Stanford University. Ph.D. in Linguistics.
Dissertation title: {\em Ergativity: Argument Structure and Grammatical
Relations.} 1994.
% Committee: Joan Bresnan (chair), Mary Dalrymple, Ivan Sag, Peter Sells.

\end{Professional Preparation}

\begin{Appointments}[1996--present]
\item[2012--present] Professor of Computer Science and Linguistics,
  Stanford University.
\item[2006--2012] Associate Professor of Computer Science and
Linguistics, Stanford University.
\item[1999--2006] Assistant Professor of Computer Science and
Linguistics, Stanford University.
\item[1996--1999] Lecturer B [$\approx$ Asst Prof.], % (tenured from 1998),
Department of Linguistics, University of Sydney. 
\item[1994--1996] Assistant Professor,
% (tenure-track),
Computational Linguistics
Program, Department of Philosophy, Carnegie Mellon University.
\end{Appointments}

\begin{Products (most closely related)}

\item Drew A. Hudson and Christopher D. Manning. 2018. 
Compositional attention networks for machine reasoning
In {\em Proceedings of the International Conference on Learning
  Representations (ICLR 2018)}.
\url{https://arxiv.org/pdf/1803.03067}

\item Abigail See, Peter J. Liu, and Christopher D. Manning. 2017. Get
  to the point: Summarization with pointer-generator networks. In
  \emph{Proceedings of ACL 2017}, pp. 1073--1083.
\url{https://nlp.stanford.edu/pubs/see2017get.pdf}

\item Yuhao Zhang, Victor Zhong, Danqi Chen, Gabor Angeli, and Christopher D. Manning. 2017.
Position-aware Attention and Supervised Data Improve Slot Filling.
In \emph{Empirical Methods in Natural Language Processing (EMNLP 2017)}. 
\textbf{Outstanding paper award.}
 \url{https://nlp.stanford.edu/pubs/zhang2017tacred.pdf}

\item Mihail Eric, Lakshmi Krishnan, Francois Charette, and Christopher~D. Manning.
2017.
Key-value retrieval networks for task-oriented dialogue.
In {\em Proceedings of the 18th Annual SIGdial Meeting on Discourse and Dialogue}, pp. 37--49.
\url{https://nlp.stanford.edu/pubs/eric2017kvret.pdf}

% \item Timothy Dozat and Christopher~D. Manning.
% 2017.
% Deep biaffine attention for neural dependency parsing.
% In \emph{Proceedings of the International Conference on Learning
%   Representations (ICLR 2017)}. 

\item Sida I. Wang, Percy Liang and Christopher D. Manning. 2016.
Learning Language Games through Interaction.
In \emph{ACL 2016}. 
\url{http://nlp.stanford.edu/pubs/wang2016games.pdf}

% \item Danqi Chen, Jason Bolton, and Christopher D. Manning. 2016. A
%  thorough examination of the CNN\slash Daily Mail reading
%   comprehension task. In \emph{Proceedings of ACL 2016}, pp. 2358--2367.
% \url{https://nlp.stanford.edu/pubs/chen2016thorough.pdf}

% \item Richard Socher, Danqi Chen, Christopher D. Manning and Andrew
%  Y. Ng. 2013. Reasoning With Neural Tensor Networks For Knowledge Base Completion.
% In \emph{Advances in Neural Information Processing Systems 26}. \url{http://nlp.stanford.edu/pubs/SocherChenManningNg\_NIPS2013.pdf}

% \item Richard Socher, Brody Huval, Christopher~D. Manning, and Andrew~Y. Ng.
% Semantic compositionality through recursive matrix-vector spaces.
% In \emph{Proceedings of the 2012 Joint Conference on Empirical
%   Methods in Natural Language Processing and Computational Natural Language
%   Learning}, pages 1201--1211, 2012.

% \item Surdeanu, Mihai  and  Tibshirani, Julie  and  Nallapati, Ramesh
%   and  Manning, Christopher D. 2012.
% Multi-instance Multi-label Learning for Relation Extraction.
% In \emph{Proceedings of the 2012 Joint Conference on Empirical Methods
%   in Natural Language Processing and Computational Natural Language
%   Learning}, pages 455--465.

% \item Richard Socher, Jeffrey Pennington, Eric~H. Huang, Andrew~Y. Ng, and
%   Christopher~D. Manning.
% {Semi-Supervised Recursive Autoencoders for Predicting Sentiment
%  Distributions}.
% In \emph{Proceedings of the 2011 Conference on Empirical Methods in
%  Natural Language Processing (EMNLP)}, pages 151--161, 2011.

% \item Richard Socher, Cliff Chiung-Yu Lin, Andrew Y. Ng, and
%   Christopher D. Manning. 2011. Parsing Natural Scenes and Natural Language with Recursive Neural Networks.
% In \emph{Proceedings of the 26th International Conference on Machine
%   Learning (ICML)}.

% \item  Sharon Goldwater, Dan Jurafsky, and Christopher
%   D. Manning. 2010. Which words are hard to recognize? Prosodic,
%  lexical, and disfluency factors that increase speech recognition
 % error rates. \emph{Speech Communication} 52: 181-200.

%\item
% Daniel Ramage, Evan Rosen, Jason Chuang, Christopher D. Manning, and
% Daniel A. McFarland. 2009. Topic Modeling for the Social Sciences. In
% \emph{NIPS 2009 Workshop on Applications for Topic Models: Text and Beyond}.

% \item
% Daniel Ramage, David Hall, Ramesh Nallapati, and Christopher
% D. Manning. 2009. Labeled LDA: A supervised topic model for credit
% attribution in multi-labeled corpora. In \emph{Proceedings of the 2009
% Conference on Empirical Methods in Natural Language Processing}.

% \item 
% Jenny Rose Finkel and Christopher D. Manning. 2009. Hierarchical
% Bayesian Domain Adaptation. In \emph{Proceedings of the North American
% Association of Computational Linguistics (NAACL 2009)}.

% \item MacCartney, Bill  and  Manning, Christopher D. 2009. An extended
%  model of natural logic. \emph{Proceedings of the Eight International
%    Conference on Computational Semantics}, pp.~140--156.

% \item Marie-Catherine de Marneffe and Anna N. Rafferty and Christopher
%   D. Manning. 2008. Finding Contradictions in Text.
% In \emph{Proceedings of the 46th Annual Meeting of the Association for Computational Linguistics (ACL 2008)},
% pages 1039--1047.

% \item Bill MacCartney, Trond Grenager, Marie-Catherine de Marneffe, Daniel Cer, and Christopher D. Manning. 2006. Learning to recognize features of valid textual entailments. \emph{HLT-NAACL 2006}, pp. 41--48. \url{http://nlp.stanford.edu/\texttildelow manning/papers/rte-naacl06.pdf}

% \item
% David L.W. Hall, Daniel Jurafsky, and Christopher
% D. Manning. 2008. Studying the History of Ideas Using Topic Models. In
% \emph{Proceedings of EMNLP 2008}.

% \item Jenny Rose Finkel, Trond Grenager and Christopher D. Manning. 2007. The Infinite Tree. \emph{Proceedings of the 45th Annual Meeting of the Association for Computational Linguistics (ACL 2007),} pp. 272--279. \url{http://nlp.stanford.edu/\texttildelow manning/papers/unsupdep.pdf}

% \item Nick Chater and Christopher D. Manning. 2006. Probabilistic models of language processing and acquisition. \emph{TRENDS in Cognitive Sciences,} 10(7): 335--344. 

% \item Trond Grenager and Christopher D. Manning. 2006. Unsupervised Discovery of a Statistical Verb Lexicon. \emph{Conference on Empirical Methods in Natural Language Processing (EMNLP 2006)}, pp. 1--8. \url{http://nlp.stanford.edu/\texttildelow manning/papers/verblex.pdf}

% \item Kristina Toutanova, Aria Haghighi, and Christopher D. Manning. 2005. Joint Learning Improves Semantic Role Labeling. \emph{Proceedings of the 43rd Annual Meeting of the Association for Computational Linguistics (ACL 2005)}, pp. 589--596. \url{http://nlp.stanford.edu/\texttildelow manning/papers/acl2005-srl.pdf}

% \item  Jenny Rose Finkel, Trond Grenager, and Christopher Manning. 2005. Incorporating Non-local Information into Information Extraction Systems by Gibbs Sampling. Proceedings of the 43rd Annual Meeting of the Association for Computational Linguistics (ACL 2005), pp. 363-370. http://nlp.stanford.edu/~manning/papers/gibbscrf3.pdf

% \item Trond Grenager, Dan Klein, and Christopher D. Manning. 2005. Unsupervised Learning of Field Segmentation Models for Information Extraction. \emph{Proceedings of the 43rd Annual Meeting of the Association for Computational Linguistics (ACL 2005)}, pp. 371-378. \url{http://nlp.stanford.edu/\texttildelow manning/papers/unsupie\_final.pdf}

% \item Kristina Toutanova, Christopher D. Manning, and Andrew
% Y. Ng. 2004. Learning Random Walk Models for Inducing Word Dependency
% Distributions. Proceedings of the 21st International Conference on
% Machine Learning (ICML 2004).
% \url{http://nlp.stanford.edu/~manning/papers/ICML2004-ppwalks.pdf}

% \item Galen Andrew, Trond Grenager, and Christopher Manning. 2004. Verb Sense and Subcategorization: Using Joint Inference to Improve Performance on Complementary Tasks. EMNLP 2004, pp. 150-157. http://nlp.stanford.edu/~manning/papers/synsense.pdf

% \item 
% Dan Klein and Christopher D. Manning. 2004. Corpus-Based Induction of Syntactic Structure: Models of Dependency and Constituency. \emph{Proceedings of the 42nd Annual Meeting of the Association for Computational Linguistics (ACL 2004)}, pp. 479--486. \url{http://nlp.stanford.edu/\texttildelow manning/papers/factored-induction-camera.pdf}

% \item 
% Christopher D. Manning. 2003. Probabilistic Syntax. In Rens Bod, Jennifer Hay, and Stefanie Jannedy (eds), \emph{Probabilistic Linguistics,} pp. 289--341. Cambridge, MA: MIT Press.
\end{Products (most closely related)}


\begin{Products (other significant)}

% \itemsep=2pt plus 2pt minus 1pt

\item Jeffrey Pennington, Richard Socher and Christopher D. Manning. 2014.
GloVe: Global Vectors for Word Representation. In \emph{EMNLP 2014}. \url{http://nlp.stanford.edu/pubs/glove.pdf}

\item Richard Socher, Alex Perelygin, Jean Wu,  Jason Chuang,
  Christopher Manning, Andrew Ng and Christopher Potts. 2013.
Recursive Deep Models for Semantic Compositionality Over a Sentiment
Treebank. In \emph{EMNLP 2013}. \url{http://nlp.stanford.edu/pubs/SocherEtAl\_EMNLP2013.pdf}

\item Christopher Manning, Prabhakar Raghavan, and Hinrich Sch\"utze. 2008. \emph{Introduction to Information Retrieval}.  Cambridge University Press.
\url{http://informationretrieval.org/}

\item Dan Klein and Christopher D. Manning. 2003. Accurate Unlexicalized Parsing. \emph{ACL 2003}, pp. 423--430. \textbf{Best paper award.} \url{http://nlp.stanford.edu/\texttildelow manning/papers/unlexicalized-parsing.pdf}

\item Christopher D. Manning and Hinrich Sch\"utze. 1999. {\em Foundations
of Statistical Natural Language Processing}.  Cambridge, MA: MIT Press. \url{https://nlp.stanford.edu/fsnlp/}

% \item Marie-Catherine de Marneffe, Bill MacCartney, and Christopher D. Manning. 2006. Generating Typed Dependency Parses from Phrase Structure Parses. \emph{LREC 2006}, pp. 449-454. \url{http://nlp.stanford.edu/\texttildelow manning/papers/LREC\_2.pdf}


% \item Christopher D. Manning. 2003. Probabilistic Syntax. In Rens Bod,
% Jennifer Hay, and Stefanie Jannedy (eds), \emph{Probabilistic Linguistics,}
% pp. 289--341. Cambridge, MA: MIT Press.

% \newpage

% \item Sepandar Kamvar, Taher Haveliwala, Christopher~D. Manning, and Gene~H.
%  Golub. 2003. Extrapolation methods for accelerating pagerank computations.
% \emph{Proceedings of The Twelfth International World Wide Web   Conference (WWW 2003)}, pp. 261--270.
% \emph{WWW 2003}, pp.~261--270.


% \item Dan Klein and Christopher D. Manning. 2003. Fast Exact Inference
% with a Factored Model for Natural Language Parsing.   
% In \emph{Advances in Neural Information Processing Systems 15} (NIPS). 
% \url{http://www-nlp.stanford.edu/\texttildelow manning/papers/lex-parser.pdf}

% \item Dan Klein and Christopher D. Manning. 2002.
% A Generative Constituent-Context Model for Improved Grammar Induction.
% \emph{Proceedings of the 40th Annual Meeting of the
% ACL}, pp.~128--135.
% \url{http://www-nlp.stanford.edu/\texttildelow manning/papers/KleinManningACL2002.pdf}

% \item Christopher D. Manning and Bob Carpenter. 2000. Probabilistic
% Parsing Using 
% Left Corner Language Models. In Harry Bunt and Anton Nijholt (eds),
% \emph{Advances in Probabilistic and Other Parsing Technologies}. Kluwer
% Academic Publishers, pp. 105--124.

% \item Dan Klein and Christopher D. Manning. 2002. Conditional Structure
%versus Conditional Estimation in NLP Models.  \emph{2002 Conference on
%Empirical 
%Methods in Natural Language Processing (EMNLP 2002)}, pp.~9--16.
%\url{http://www-nlp.stanford.edu/\texttildelow manning/papers/objective-functions.pdf}

%\item Avery D. Andrews and Christopher D. Manning. 1999. {\em Complex
%Predicates and Information Spreading in LFG}. 
%Stanford, CA: CSLI Publications.

%\item Dan Klein, Sepandar D. Kamvar, and Christopher
%D. Manning. 2002. From Instance-level Constraints to Space-level
%Constraints: Making the Most of Prior Knowledge in Data Clustering.
%\emph{Proceedings of the Nineteen International Conference on
%Machine Learning (ICML 19)}, pp.~307--313.
%\url{http://www-nlp.stanford.edu/\texttildelow manning/papers/constrained\_clustering.ps}

%\item Toutanova, Kristina and Christopher D. Manning. 2000. Enriching the
%Knowledge Sources Used in a Maximum Entropy Part-of-Speech
%Tagger. \emph{Joint SIGDAT Conference on Empirical
%Methods in Natural Language Processing and Very Large Corpora
%(EMNLP/VLC-2000),} Hong Kong, pp. 63--70.
%\url{http://nlp.stanford.edu/\texttildelow manning/papers/emnlp2000.ps}

% \item Christopher D. Manning, Ivan A. Sag, and Masayo Iida. 1999. The
% lexical integrity of Japanese causatives.
% In Robert Levine and Georgia Green (eds), \emph{Studies in
% Contemporary Phrase Structure Grammar}. 
% Cambridge: Cambridge University Press, pp. 39--79.

% \item Manning, Christopher D. and Ivan A. Sag. 1998. Argument
% Structure, Valence, and Binding. 
% \emph{Nordic Journal of Linguistics} 21(2): 107--144.

% \item Christopher D. Manning. 1996. {\em Ergativity: Argument
% structure and grammatical relations}. Stanford, CA: CSLI Publications.

% \item Christopher D. Manning. 1993. Automatic acquisition of a large subcategorization dictionary 
% from corpora.  In the {\em Proceedings of the 31st Annual Meeting of the {ACL}}, pp.~235--242.
% \url{http://nlp.stanford.edu/\texttildelow manning/papers/subcats.pdf}
\end{Products (other significant)}

\begin{Synergistic Activities}
% \itemsep=2pt plus 2pt minus 1pt

\item Cowrote leading textbooks in Natural
  Language Processing and Information Retrieval.

\item Tutorials: Taught Deep Learning tutorial at Joint Statistical Meetings 2018 (with Ruslan Salakhutdinov),
Neural Machine Translation tutorial at ACL 2016 (with Kyunghyun Cho and Thang Luong), and 
Deep Learning for NLP tutorials at ACL 2012 and NAACL 2013 (with Yoshua Bengio and Richard Socher).
Taught earlier NLP tutorials at AAAI 2000, NIPS 2001, NAACL 2003, ACL 2003, and Digital Humanities 2011.

\item Developed and taught one of the first massively open online
  courses (MOOCs): Dan Jurafsky and Christopher D. Manning. Natural Language Processing. 70,000 students, Winter 2012.

\item Distribute Stanford CoreNLP, a widely used set of open source NLP tools for statistical parsing,
POS tagging, named entity recognition, coreference, word segmentation, relation extraction, etc., supporting English, Chinese, Spanish, German, French, and Arabic.
% tied together as Stanford CoreNLP.
% maxent classification, and tree matching, tied together as a
% complete Stanford CoreNLP pipeline.

\item Give a class on NLP at Stanford ai4all/SAILORS, a summer program for rising 10th grade girls to get them 
interested and involved in AI research. \url{http://ai4all.stanford.edu/}

% our lexicalized statistical
% natural language parsing program, a large-scale conditional loglinear
% model classifier, and various other code.

% \item Current member of the Linguistic Society of America Committee on
%  Endangered Languages;
% and their Preservation; past
% student representative on
%% the
%  Linguistic Society of America Executive Committee,
% (Bernard and Julia Bloch Memorial Fellow), 
% 1993--1995. %, including
% membership of the LSA Interim Committee on the Status of Minorities
% in Linguistics. 1993--94.

%\item Mentor TA.  Stanford University Linguistics Dept, 1993--94. 
%Completed the first edition of a handbook for Stanford linguistics
%TAs, and organized a workshop for linguistics TA training.

% \item Editorial board for \emph{Computational Linguistics}
\end{Synergistic Activities}


\vspace*{24pt}

\vfill
\end{vita}
\end{document}

\endinput



\begin{Collaborators and Other Affiliations}
\item {\bf Collaborators:} 
{\footnotesize
Eneko Agirre (U. Basque Country), % 2016
% Nicholas Andrews (JHU), % 2012
Gabor Angeli (Eloquent Labs), % 2016
Steven Bethard (U. Arizona), % 2014
Jonathan Berant (Stanford), % 2013
John Bauer (Vicarious AI), % 2014
% Bharath Bhat (Diffbot),
Jason Bolton (Stanfrd), % 2016
Samuel Bowman (NYU), % 2016
Daniel Cer (Stanford), % 2014
Arun Chaganty (Stanford), % 2016
% Pichuan Chang (AltSchool),
Angel Chang (Princeton), % 2016
Wanxiang Che (Harbin IT), % 2013
Danqi Chen (Stanford), % 2016
Pei-Chun Chen (Google), % 2014
Jason Chuang (U. Washington), % 2014
Kevin Clark (Stanford), % 2016
Peter Clark (Allen Institute for AI), % 2014
Miriam Connor (Google), % 2014
Timothy Dozat (Stanford), % 2014
% Mark Dredze (JHU),
% Susan Dumais (Microsoft),
Jenny Finkel (MixPanel), % 2014
% Dan Flickinger (Stanford),
Roy Frostig (Stanford), % 2014
% Michel Galley (Microsoft),
Milind Ganjoo (Twitter), % 2013
Jon Gauthier (Stanford), % 2016
% Niyu Ge (IBM),
Filip Ginter (U. Turku), % 2016
Yoav Goldberg (Bar Ilan), % 2016
% Sharon Goldwater (Stanford),
% Matthew Gormley (JHU),
% Gene Golub (Stanford),
Spence Green (Stanford), % 2014
Raghav Gupta (Stanford), % 2016
Sonal Gupta (Stanford), % 2014
% Andrey Gusev (Sift Science),
Jan Hajic (Prague), % 2016
Brittany Harding (U. Washington), % 2013
Katri Haverinen (U. Turku), % 2014
Jeffrey Heer (U. Washington), % 2015
% Eric Huang (Stanford), 2012
% Brody Huval (Stanford), 2012
% Trond Grenager (Stanford),
% Claire Grover (Edinburgh),
% Aria Haghighi (UC Berkeley),
% David Hall (Stanford),
% Trevor Hastie (Stanford),
% Taher Haveliwala (Google),
Kenneth Heafield (Edinburgh), % 2014
% Masayo Iida (InXight Software),
% Tolga Ilhan (Stanford),
% Nitin Indurkhya (Nanyang Technical U.),
% Kevin Jansz (Macquarie Bank),
Daniel Jurafsky (Stanford), % 2014
% Sepandar Kamvar (Google),
Andrej Karpathy (Open AI), % 2014
% Chlo\'e Kiddon (Stanford),
% Dan Klein (UC Berkeley),
% Ewan Klein (Edinburgh),
% Daphne Koller (Stanford),
Ranjay Krishna (Stanford), % 2015
% Vijay Krishnan (Stanford),
% Pat Langley (Arizona State),
% Iddo Lev (Stanford),
% Roger Levy (UC San Diego),
Michael Kayser (Stanford), % 2015
Quoc Le (Google), % 2014
Anna Lembke (Stanford), % 2015
Justin Lewis (U. Washington), % 2013
Fei-Fei Li (Stanford), % 2015
Percy Liang (Stanford), % 2016
% Cliff Chiung-Yu Lin (Google),
Ting Liu (Harbin IT), % 2013
Thang Minh Luong (Google), % 2016
% Andrew Maas (Stanford),
% Bill MacCartney (Google),
Diana MacLean (Stanford), % 2015
% Penka Markova (Stanford),
Marie-Catherine de Marneffe (Ohio State), % 2016
% Andrew McCallum (U. Mass.),
David McClosky (IBM), % 2014
Ryan McDonald (Google), % 2016
% Daniel A. McFarland (Stanford),
William Monroe (Stanford), % 2015
% Robert Munro (Idibon), 2012
% John T. Maxwell III (Xerox PARC),
% Mark Mitchell (San Jose State),
% Huy Nguyen (Plaxo),
% Jamie Nicolson (Google),
% Malvina Nissim (Edinburgh),
% Ramesh Nallapati (IBM),
Neha Nayak (Google), % 2016
Julia Neidert (Google), % 2014
Joakim Nivre (Uppsala), % 2016
Andrew Ng (Stanford), % 2014
% Stephan Oepen (Stanford),
% Satoshi Oyama (Kyoto),
% Stuart Shieber (Harvard),
% Sebastian Pado (Heidelberg),
Jeffrey Pennington (Google), % 2014
Alex Perelygin (Stanford), % 2013
Slav Petrov (Google), % 2016
Hieu Pham (CMU), % 2015
Christopher Potts (Stanford), % 2016
Melvin Johnson Premkumar (Stanford), % 2015
% Susan Poetsch (Macquarie),
% David M. W. Powers (Flinders U.)
Sampo Pyysalo (Cambridge), % 2016
% Anna Rafferty (Carleton College),
% Prabhakar Raghavan (Yahoo!\ Research),
% Rajat Raina (Stanford),
% Daniel Ramage (Google),
Abhinav Rastogi (Stanford), % 2016
Christopher R\'e (Stanford), % 2015
Kevin Reschke (Stanford), % 2014
% Sebastian Riedel (UC London),
% Ivan A. Sag (Stanford),
Manolis Savva (Princeton), % 2015
Sebastian Schuster (Stanford), % 2016
% Hinrich Sch\"utze (Stuttgart),
Abigail See (Stanford), % 2016
% Stuart Shieber (Harvard),
Natalia Silveira (Apple), % 2016
% Jane Simpson (U. Sydney),
% Gail Sinclair (U. Edinburgh),
% Yoram Singer (Google),
% Joseph Smarr (Stanford)
% Wee Jim Sng (Nanyang Tech. U.),
Valentin Spitkovsky (Google), % 2016
Vivek Srikumar (U. Utah), % 2014
Aju Thalappilllil Scaria (Google), % 2013
Natalia Silveira (Stanford), % 2016
Richard Socher (Metamind), % 2015
Mihai Surdeanu (U. Arizona), % 2014
Kai Sheng Tai (Stanford), % 2015
Julie Tibshirani (Palantir), % 2015
% Ben Taskar (Penn),
% Cynthia A. Thompson (U. Utah),
% Kristina Toutanova (Microsoft),
% Huihsin Tseng (Stanford),
Reut Tsarfaty (Weizmann), % 2016
% Shivakumar Vaithyanathan (IBM),
Rob Voigt (Stanford), % 2014
Stefan Wager (Stanford), % 2013
Mengqiu Wang (Stanford), % 2014
Sida Wang (Stanford), % 2016
Keenon Werling (Eloquent Labs), % 2015
Jean Wu (Apple), % 2014
Sen Wu (Stanford), % 2014
Daniel Zeman (Prague), % 2016
Ce Zhang (Stanford), % 2014
Will Zou (Stanford). % 2013
% Eric Yeh (Stanford).

}

% List for 2008-09
% Beatrice Alex (Edinburgh),
% Galen Andrew (Microsoft Research),
% Avery D. Andrews (Australian National U.),
% I Wayan Arka (Udayana U., Bali), 
% Brett Baker (U. Sydney),
% Philip Beineke (Stanford),
% Thorsten Brants (PARC Inc.),
% Joan Bresnan (Stanford),
% Bob Carpenter (Lucent Bell Labs),
% Nick Chater (UC London),
% Ann Copestake (Stanford),
% Nathanael Chambers (Stanford),
% Michael Collins (MIT),
% Miriam Corris (Sydney),
% Christopher Cox (Facebook),
% Ofer Dekel (Hebrew University),
% Shipra Dingare (Edinburgh),
% Nick J. Enfield (U. Melbourne),


% List for 2005-2008
% Beatrice Alex (Edinburgh U.),
% Galen Andrew (Microsoft Research),
% Daniel Cer (U. Colorado, Boulder),
% Nick Chater (UC London),
% Nathanael Chambers (Stanford),
% Pichuan Chang (Stanford),
% Christopher Cox (Facebook),
% Shipra Dingare (Edinburgh),
% Jenny Finkel (Stanford),
% Dan Flickinger (Stanford),
% Michel Galley (Stanford)
% Sharon Goldwater (U. Edinburgh)
% Trond Grenager (adap.tv),
% Claire Grover (Edinburgh),
% Aria Haghighi (UC Berkeley),
% David Hall (Stanford),
% Trevor Hastie (Stanford),
% Daniel Jurafsky (Stanford),
% Chloe Kiddon (U. Washington),
% Alex Kleeman (Stanford),
% Dan Klein (UC Berkeley),
% Vijay Krishnan (Stanford),
% Pat Langley (Arizona State),
% Bill MacCartney (Stanford),
% Marie%Catherine de Marneffe (Stanford),
% Jeff Michels (Stanford),
% Jamie Nicolson (Google),
% Malvina Nissim (Edinburgh),
% Andrew Ng (Stanford),
% Stephan Oepen (U. of Oslo),
% Anna Rafferty (UC Berkeley),
% Prabhakar Raghavan (Yahoo! Research),
% Rajat Raina (Stanford),
% Daniel Ramage (Stanford),
% Hinrich Schuetze (U. Stuttgart),
% Yun%Hsuan Sung (Stanford) 
% Kristina Toutanova (Microsoft Research),
% Huihsin Tseng (Yahoo!),
% Eric Yeh (SRI).


\item {\bf Graduate advisor:} Joan Bresnan (Stanford). [No
  postdoctoral appointment.]

\item {\bf Thesis Advisor and Postgraduate-Scholar Sponsor:} 

{\footnotesize
PhD advisor for 27, 20 completed: Lynn Berry (Appen), Sepandar Kamvar
(MIT); Dan Klein (UC Berkeley), Kristina Toutanova (Microsoft
Research), Roger Levy (MIT), Bill
MacCartney (Apple), Pi-Chuan Chang (Google), Jenny Finkel (MixPanel), Dan Ramage (Google),
Marie-Catherine de Marneffe (Ohio State), Robert Munro (Idibon),
Jason Chuang (Washington), 
Richard Socher (Metamind), Mengqiu Wang (startup in China),
Spence Green (Lilt), Sonal Gupta (Viv Labs), Angel Chang (Princeton),
Gabor Angeli (Eloquent Labs), Samuel Bowman (NYU), Natalia Silveira (Apple);
and 8 ongoing at Stanford:
Thang Luong Minh, Sida Wang, Danqi Chen, Kevin Clark, Peng Qi, Abigail
See, Tim Dozat, and Matt Lamm.

Postdoctoral advisor for 6 completed: Jonathan Berant (Tel Aviv), Michel Galley
(Microsoft Research), Kenneth Heafield (Edinburgh), David McClosky
(IBM Research), Jeffrey Pennington (Google), and Vivek Srikumar (U. Utah).

}
\end{Collaborators and Other Affiliations}


\item Kristina Toutanova, Mark Mitchell and Christopher
Manning. 2003. Optimizing Local Probability Models for Statistical
Parsing. In Proceedings of 14th European Conference on Machine Learning
(ECML 2003), pp.~409--420.
\url{http://www-nlp.stanford.edu/\texttildelow manning/papers/experiments3rd.pdf}

\item Dan Klein, Joseph Smarr, Huy Nguyen, and Christopher D. Manning. 2003. Named Entity Recognition with Character-Level Models. Proceedings the Seventh Conference on Natural Language Learning, pp. 180-183. 
\url{http://www-nlp.stanford.edu/\texttildelow manning/papers/conll-ner.pdf}
