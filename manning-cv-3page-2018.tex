\documentclass[11pt]{article}

\usepackage{palatino}
\usepackage{cv}
\usepackage{textcomp}

\textheight=9.1in
\textwidth=6.5in
\oddsidemargin=0in
\advance\topmargin -0.4in

\def\url#1{{\small\sf #1}}
% \def\enumsquare{\raise1pt\hbox{$\scriptstyle\blacksquare$}}
% \renewcommand{\labelitemi}{\enumsquare}

%\lines
%\refsonrequest
%\dated
\centername
\namesize{\Large}
\homeaddresstitlefalse

\begin{document}
\vspace*{-0.75cm}

\name{\bf Christopher David Manning}
\businessaddress{Department of Computer Science\\
        Gates Building 2A, 353 Serra Mall\\
        Stanford CA 94305-9020\\
        USA}

\homeaddress{Phone: +1 (650) 723-7683\\
	Fax:  +1 (650) 725-1449 \\
	Email: \small\sf manning@cs.stanford.edu \\
        Web: \small\sf http://nlp.stanford.edu/\texttildelow manning/}

\newdatedcategory{Professional Preparation}
\newdatedcategory{Appointments}
\newcategory{Collaborators and Other Affiliations}
\newcategory{Important Recent Publications}
\newcategory{Other Key Publications}
\newcategory{Synopsis}


\begin{vita}
\begin{Professional Preparation}[1990--1994]
\item[1990--1994] Ph.D. Stanford University, Dept of
Linguistics, awarded January 1995.
Dissertation title: {\em Ergativity: Argument Structure and Grammatical
Relations.}
% Committee: Joan Bresnan (chair), Mary Dalrymple, Ivan Sag, Peter Sells.
\item[1984--1989] B.A.\ (Hons) with First Class Honors and University
Medal in Linguistics,\\
The Australian National University.
% Additional majors in Mathematics and Computer Science.
\end{Professional Preparation}

\begin{Appointments}[1996--present]
\item[2016--present] Thomas M. Siebel Professor in Machine Learning, Professor of Linguistics and of Computer Science, Stanford University.
\item[2012--2016] Professor of Computer Science and Linguistics,
  Stanford University.
\item[2006--2012] Associate Professor of Computer Science and
Linguistics, Stanford University.
\item[1999--2006] Assistant Professor of Computer Science and
Linguistics, Stanford University.
\item[1996--1999] Lecturer B [$\approx$ Asst Prof.], % (tenured from 1998),
Department of Linguistics, University of Sydney. 
\item[1994--1996] Assistant Professor,
% (tenure-track),
Computational Linguistics,
% Program, 
% Department of Philosophy, 
Carnegie Mellon University.
\end{Appointments}

\begin{Synopsis}
\item
Manning works on systems for natural language understanding.
Following fifteen years of pioneering work in exploring probabilistic models
of Natural Language Processing, he is now a leader in
applying Deep Learning to Natural Language Processing, with well-known
research on Tree Recursive Neural Networks, sentiment analysis, neural
network dependency parsing, the GloVe model of word vectors, neural
machine translation, question answering, natural language inference,
and other deep language understanding. Manning has
coauthored leading textbooks on statistical approaches to Natural
Language Processing (Manning and Sch\"utze 1999) and information
retrieval (Manning, Raghavan, and Sch\"utze, 2008), is the lead
developer of the widely used Stanford CoreNLP software, and is an ACM Fellow,
a AAAI Fellow, an ACL Fellow, and Past President of the
ACL\@. Research of his has won Best Paper Awards at ACL, EMNLP, Coling, and CHI. 
\end{Synopsis}

\begin{Important Recent Publications}

\item Drew A. Hudson and Christopher D. Manning. 2018. 
Compositional attention networks for machine reasoning
In {\em Proceedings of the International Conference on Learning
  Representations (ICLR 2018)}.

\item Abigail See, Peter J. Liu, and Christopher D. Manning. 2017. Get
  to the point: Summarization with pointer-generator networks. In
  \emph{Proceedings of ACL 2017}, pp. 1073--1083.

\item Timothy Dozat and Christopher~D. Manning.
2017.
Deep biaffine attention for neural dependency parsing.
In {\em Proceedings of the International Conference on Learning
  Representations (ICLR 2017)}.

% \item Timothy Dozat, Peng Qi, and Christopher~D. Manning. 2017. Stanford's graph-based neural dependency parser at the {CoNLL} 2017
%  shared task. In {\em Proceedings of the CoNLL 2017 Shared Task: Multilingual                                                                             
%   Parsing from Raw Text to Universal Dependencies}, pp. 20--30.

% \item Kevin Clark and Christopher~D. Manning. 2016. Deep reinforcement learning for mention-ranking coreference models.
% In {\em EMNLP 2017}, pp. 2256--2262.

\item Danqi Chen, Jason Bolton, and Christopher D. Manning. 2016. A
  thorough examination of the CNN\slash Daily Mail reading
  comprehension task. In \emph{Proceedings of ACL 2016}, pp. 2358--2367.

\item Minh-Thang Luong and Christopher~D. Manning. 2016. Achieving open vocabulary neural machine translation with hybrid
  word-character models. In {\em ACL 2016}, pp. 1054--1063.

\item Samuel~R. Bowman, Jon Gauthier, Abhinav Rastogi, Raghav Gupta, Christopher~D.
  Manning, and Christopher Potts.
2016.
A fast unified model for parsing and sentence understanding.
In {\em Proceedings of the 54th Annual Meeting of the Association for
  Computational Linguistics (Volume 1: Long Papers)}, pp. 1466--1477.

\item Samuel R. Bowman, Gabor Angeli, Christopher Potts, and Christopher D. Manning. 2015. A large annotated corpus for learning natural language inference. In \emph{Proceedings of the 2015 Conference on Empirical Methods in Natural Language Processing}. Best new data set award.

\item Kai Sheng Tai, Richard Socher, and Christopher
  D. Manning. 2015. Improved Semantic Representations From
  Tree-Structured Long Short-Term Memory Networks. In
  \emph{ACL 2015.}

\item Thang Luong, Hieu Pham, and Christopher~D. Manning. 2015. Effective approaches to attention-based neural machine translation.
\newblock In {\em Proceedings of the 2015 Conference on Empirical Methods in                                                                          
 Natural Language Processing}, pp. 1412--1421.

\item Richard Socher, Andrej Karpathy, Quoc~V. Le, Christopher~D. Manning, and
  Andrew~Y. Ng.
2014.
Grounded compositional semantics for finding and describing images
  with sentences.
{\em Transactions of the Association for Computational Linguistics}
  pp. 207--218.

\item Jeffrey Pennington, Richard Socher and Christopher D. Manning. 2014.
GloVe: Global Vectors for Word Representation. In \emph{EMNLP 2014}. \url{http://nlp.stanford.edu/pubs/glove.pdf}

\item Danqi Chen and Christopher~D Manning.
2014.
A fast and accurate dependency parser using neural networks.
In {\em Empirical Methods in Natural Language Processing (EMNLP 2014)}.

\end{Important Recent Publications}


\begin{Other Key Publications}

\item Christopher~D. Manning, Mihai Surdeanu, John Bauer, Jenny Finkel, Steven~J.
  Bethard, and David McClosky.
2014.
The {Stanford} {CoreNLP} natural language processing toolkit.
In {\em Proceedings of 52nd Annual Meeting of the Association for
  Computational Linguistics: System Demonstrations}, pp. 55--60.

\item Richard Socher, Alex Perelygin, Jean Wu,  Jason Chuang,
  Christopher Manning, Andrew Ng and Christopher Potts. 2013.
Recursive Deep Models for Semantic Compositionality Over a Sentiment
Treebank. In \emph{EMNLP 2013}. \url{http://nlp.stanford.edu/pubs/SocherEtAl\_EMNLP2013.pdf}


\item Christopher Manning, Prabhakar Raghavan, and Hinrich Sch\"utze. 2008. \emph{Introduction to Information Retrieval}.  Cambridge University Press.
\url{http://informationretrieval.org/}

\item Marie-Catherine de Marneffe, Bill MacCartney, and Christopher D. Manning. 2006. Generating Typed Dependency Parses from Phrase Structure Parses. \emph{LREC 2006}, pp. 449-454. \url{http://nlp.stanford.edu/\texttildelow manning/papers/LREC\_2.pdf}

\item  Jenny Rose Finkel, Trond Grenager, and Christopher Manning. 2005. Incorporating Non-local Information into Information Extraction Systems by Gibbs Sampling. Proceedings of the 43rd Annual Meeting of the Association for Computational Linguistics (ACL 2005), pp. 363-370. \url{http://nlp.stanford.edu/\texttildelow manning/papers/gibbscrf3.pdf}

\item 
% #7
Kristina Toutanova, Dan Klein, Christopher D. Manning, and Yoram Singer.
2003. Feature-rich part-of-speech tagging with a cyclic dependency network.
In \emph{Human Language Technology Conference of the North American Chapter
of the Association for Computational Linguistics (HLT-NAACL 2003)}, pp.
252--259.

\item Dan Klein and Christopher D. Manning. 2003. Accurate Unlexicalized Parsing. \emph{ACL 2003}, pp. 423--430. \url{http://nlp.stanford.edu/\texttildelow manning/papers/unlexicalized-parsing.pdf}


\item Christopher D. Manning and Hinrich Sch\"utze. 1999. \emph{Foundations
of Statistical Natural Language Processing}.  Cambridge, MA: MIT Press.

\item Christopher D. Manning. 1993. Automatic acquisition of a large
subcategorization dictionary 
from corpora.  In the {\em Proceedings of the 31st Annual Meeting
of the {ACL}}, pp.~235--242.
\url{http://nlp.stanford.edu/\texttildelow manning/papers/subcats.pdf}

\end{Other Key Publications}


\end{vita}
\end{document}
