\documentclass[11pt]{article}

\usepackage{palatino}
\usepackage{cv}
\usepackage{textcomp}

\textheight=9.1in
\textwidth=6.5in
\oddsidemargin=0in
\advance\topmargin -0.4in

\def\url#1{{\small\sf #1}}
% \def\enumsquare{\raise1pt\hbox{$\scriptstyle\blacksquare$}}
% \renewcommand{\labelitemi}{\enumsquare}

%\lines
%\refsonrequest
%\dated
\centername
\namesize{\Large}
\homeaddresstitlefalse

\begin{document}
\vspace*{-0.75cm}

\name{\bf Christopher David Manning}
\businessaddress{Department of Computer Science\\
        Gates Building 2A, 353 Serra Mall\\
        Stanford CA 94305-9020\\
        USA}

\homeaddress{Phone: +1 (650) 723-7683\\
	Fax:  +1 (650) 725-1449 \\
	Email: \small\sf manning@cs.stanford.edu \\
        Web: \small\sf http://nlp.stanford.edu/\texttildelow manning/}

\newdatedcategory{Professional Preparation}
\newdatedcategory{Appointments}
\newcategory{Collaborators and Other Affiliations}
\newcategory{Representative Publications}
\newcategory{Synopsis}


\begin{vita}
\begin{Professional Preparation}[1990--1994]
\item[1990--1994] Ph.D. Stanford University, Dept of
Linguistics, awarded January 1995.
Dissertation title: {\em Ergativity: Argument Structure and Grammatical
Relations.}
% Committee: Joan Bresnan (chair), Mary Dalrymple, Ivan Sag, Peter Sells.
\item[1984--1989] B.A.\ (Hons) with First Class Honors and University
Medal in Linguistics,\\
The Australian National University.
% Additional majors in Mathematics and Computer Science.
\end{Professional Preparation}

\begin{Appointments}[1996--present]
\item[2012--present] Professor of Computer Science and Linguistics,
  Stanford University.
\item[2006--2012] Associate Professor of Computer Science and
Linguistics, Stanford University.
\item[1999--2006] Assistant Professor of Computer Science and
Linguistics, Stanford University.
\item[1996--1999] Lecturer B [$\approx$ Asst Prof.], % (tenured from 1998),
Department of Linguistics, University of Sydney. 
\item[1994--1996] Assistant Professor,
% (tenure-track),
Computational Linguistics,
% Program, 
% Department of Philosophy, 
Carnegie Mellon University.
\end{Appointments}

\begin{Synopsis}
\item
Christopher works on software for natural language understanding.
 He is a leader in
applying Deep Learning to Natural Language Processing, with well-known
research on Tree Recursive Neural Networks, sentiment analysis, neural
network dependency parsing, the GloVe model of word vectors, neural
machine translation, and deep language understanding. Manning has
coauthored leading textbooks on statistical approaches to Natural
Language Processing (NLP) (Manning and Sch\"utze 1999) and information
retrieval (Manning, Raghavan, and Sch\"utze, 2008), is the lead
developer behind the Stanford CoreNLP software, and is an ACM Fellow,
a AAAI Fellow, an ACL Fellow, and Past President of the
ACL\@. Research of his has won ACL, Coling, EMNLP, and CHI Best Paper
Awards. 
\end{Synopsis}

\begin{Representative Publications}

\item Abigail See, Peter J. Liu, and Christopher D. Manning. 2017. Get
  to the point: Summarization with pointer-generator networks. In
  \emph{Proceedings of ACL 2017}, pp. 1073–1083.

\item Timothy Dozat, Peng Qi, and Christopher~D. Manning. 2017. Stanford's graph-based neural dependency parser at the {CoNLL} 2017
  shared task. In {\em Proceedings of the CoNLL 2017 Shared Task: Multilingual                                                                             
  Parsing from Raw Text to Universal Dependencies}, pp. 20--30.

% \item Minh-Thang Luong and Christopher~D. Manning. 2016. Achieving open vocabulary neural machine translation with hybrid
%   word-character models. In {\em ACL 2016}, pp. 1054--1063.

% \item Kevin Clark and Christopher~D. Manning. 2016. Deep reinforcement learning for mention-ranking coreference models.
% In {\em EMNLP 2017}, pp. 2256--2262.

\item Danqi Chen, Jason Bolton, and Christopher D. Manning. 2016. A
  thorough examination of the CNN\slash Daily Mail reading
  comprehension task. In \emph{Proceedings of ACL 2016}, pp. 2358–2367.

\item Kai Sheng Tai, Richard Socher, and Christopher
  D. Manning. 2015. Improved Semantic Representations From
  Tree-Structured Long Short-Term Memory Networks. In
  \emph{ACL 2015.}

\item Thang Luong, Hieu Pham, and Christopher~D. Manning. 2015. Effective approaches to attention-based neural machine translation.
\newblock In {\em Proceedings of the 2015 Conference on Empirical Methods in                                                                          
 Natural Language Processing}, pp. 1412--1421.

\end{Representative Publications}


\end{vita}
\end{document}
